
% Cal Poly Thesis
% 
% based on UC Thesis format
%
% modified by Mark Barry 2/07.
%

\documentclass[12pt]{ucthesis}

\usepackage{ifpdf} 
\newif\ifpdf
\ifx\pdfoutput\undefined
    \pdffalse % we are not running PDFLaTeX
\else
\pdfoutput=1 % we are running PDFLaTeX
\pdftrue \fi

\usepackage{url}
\usepackage{multicol}
\ifpdf

    \usepackage[pdftex]{graphicx}
    % Update title and author below...
    \usepackage[pdftex,plainpages=false,breaklinks=true,colorlinks=true,urlcolor=blue,citecolor=blue,%
                                       linkcolor=blue,bookmarks=true,bookmarksopen=true,%
                                       bookmarksopenlevel=3,pdfstartview=FitV,
                                       pdfauthor={Forrest Reiling},
                                       pdftitle={Extending Windowing Systems to Three Dimensions},
                                       pdfkeywords={thesis, masters, cal poly}
                                       ]{hyperref}
    %Options with pdfstartview are FitV, FitB and FitH
    \pdfcompresslevel=1

\else
    \usepackage{graphicx}
\fi

\usepackage{hyperref}
\hypersetup{
	linktoc=all,
    colorlinks,
    citecolor=black,
    filecolor=black,
    linkcolor=black,
    urlcolor=black
}



\usepackage{titlesec}
% \titleformat{\chapter}[display]% OLD
%     {\normalfont\huge\bfseries}{\chaptertitlename\ \thechapter}{20pt}{\Huge}% OLD
% \titlespacing*{\chapter}{0pt}{50pt}{40pt}% OLD
\titleformat{\chapter}[display]% NEW
    {\normalfont\centering}{\chaptertitlename\ \thechapter}{12pt}{}% NEW
\titlespacing*{\chapter}{0pt}{30pt}{20pt}% NEW

%\titleformat{\section}[block]{first}{label}{12pt}

\titleformat{\section}{}{\thesection}{1em}{}
\titleformat{\subsection}{}{\thesubsection}{1em}{}
\titleformat{\subsubsection}{}{\thesubsubsection}{1em}{}
\titleformat{\paragraph}{}{\theparagraph}{1em}{}

\usepackage[font={}]{caption}

%\renewcommand{\cftchapleader}{\cftdotfill{\cftdotsep}} % for chapters
%\renewcommand{\cftsecleader}{\cftdotfill{\cftdotsep}} 


\usepackage{amssymb}
\usepackage{amsmath}
\usepackage[letterpaper]{geometry}
\usepackage[overload]{textcase}
\usepackage[toc]{appendix}
\renewcommand\appendixtocname{Appendix}

\usepackage{tabularx}
\usepackage{algorithmicx}
\usepackage{algpseudocode}

\usepackage{enumitem}
\setlist{nolistsep}

\usepackage{float}

\floatstyle{boxed}
\restylefloat{table}

\bibliographystyle{abbrv}

\setlength{\parindent}{0.25in} \setlength{\parskip}{6pt}

\geometry{verbose,nohead,tmargin=1.25in,bmargin=1in,lmargin=1.5in,rmargin=1.3in}

\setcounter{tocdepth}{4}
\setcounter{secnumdepth}{4}

% Different font in captions (single-spaced, bold) ------------
%\newcommand{\captionfonts}{\small\bf\ssp}
%\newcommand{\captionfonts}{}

\makeatletter  % Allow the use of @ in command names
%\long\def\@makecaption#1#2{%
%  \vskip\abovecaptionskip
%  \sbox\@tempboxa{{\captionfonts #1: #2}}%
%  \ifdim \wd\@tempboxa >\hsize
%    {\captionfonts #1: #2\par}
%  \else
%    \hbox to\hsize{\hfil\box\@tempboxa\hfil}%
%  \fi
%  \vskip\belowcaptionskip}
%\makeatother   % Cancel the effect of \makeatletter
% ---------------------------------------



\begin{document}

% Declarations for Front Matter

% Update fields below!
\title{Juiciness in Citizen Science Computer Games: Analysis of a Prototypical Game}
\author{Eric Buckthal}

\degreemonth{June} \degreeyear{2014} \degree{Master of Science}
\defensemonth{May} \defenseyear{2014}

\numberofmembers{3} 
   \chair{Professor Zo{\"e} Wood, Ph.D.,\newline Department of Computer Science} 
   \othermemberA{Professor Alex Dekhtyar, Ph.D.,\newline Department of Computer Science} 
   \othermemberB{Professor Foaad Khosmood, Ph.D.,\newline Department of Computer Science} 

\field{Computer Science} \campus{San Luis Obispo}
\copyrightyears{seven}



\maketitle

\begin{frontmatter}

% Custom made for Cal Poly (by Mark Barry, modified by Andrew Tsui).
\copyrightpage

% Custom made for Cal Poly (by Andrew Tsui).
\committeemembershippage

\begin{abstract}

Incorporating the collective problem-solving skills of non-experts could accelerate the advancement of scientific research. Citizen science games leverage puzzles to present computationally difficult problems to players. Such games typically map the scientific problem to game mechanics and visual feed-back helps players improve their solutions. Like games for entertainment, citizen science games intend to capture and retain player attention. ``Juicy" game design refers to augmented visual feedback systems that give a game personality without modifying fundamental game mechanics. A ``juicy" game feels alive and polished. This thesis explores the use of ``juicy" game design applied to the citizen science genre. We present the results of a user study in its effect on player motivation with a prototypical citizen science game inspired by clustering-based E. coli bacterial strain analysis.

\end{abstract}

%\begin{acknowledgements}

%\end{acknowledgements}

\tableofcontents

\listoftables

\listoffigures

\end{frontmatter}

\pagestyle{plain}

\renewcommand{\baselinestretch}{1.66}


% ------------- Main chapters here --------------------
\chapter{Introduction}

Though the term ``gamer" still has connotations of an teenager blasting zombies in the basement, gaming has been adopted by mainstream culture \cite{sciencefriday2014}. Bejeweled, Candy Crush, and Farmville are only a few examples of games that have gained widespread acceptance outside of the traditional gamer persona. Computer and video games are a ``form of entertainment enjoyed by a diverse, worldwide consumer base that demonstrates immense energy and enthusiasm for games." \cite{softass} 59\% of American citizens play games, the average player is 31 years old, and 48\% of players are female. Puzzle, board game, game show, trivia, and card games make up 28\% of online games played \cite{softass}.

Leveraging the motivation to play games and humans' ability to recognize patterns, researchers have empowered users to perform citizen science. Examples of this include The Milky Way Project where users identify celestial bodies \cite{milkyway2014} and interactive biology applications such as Fold-it \cite{cooper2010challenge}. This partnership has introduced a more general genre of scientific discovery games which take advantage of human problem solving abilities to solve computationally difficult research problems. 

Because scientific discovery games translate these research problems into games, they rely on many fundamentals of game design including the explanation of game mechanics, the design of introductory levels, and potentially the scientific concepts. More importantly, scientific discovery games' goal is to provide an interface which non-expert players can apply knowledge in a specific scientific domain. \cite{cooper2010challenge} While fun is not the primary objective, citizen science games and other genres can enhance motivation in their game by applying traditional game design strategies. Specifically, this thesis focuses on ``juicy" game design techniques.

In this paper, we define ``juicy" design and it's importance in the genre of citizen science games and scientific discovery games. We also explore several examples of existing scientific discovery games and their ``juiciness." By prototyping and experimenting with two versions of a scientific discovery game, we conclude that ``juiciness" in citizen science games decreases perceived difficulty, increases understanding of fundamental game concepts, and improves enjoyability of citizen science games. We discuss the potential for ``juiciness" to improve player motivation and efficency at citizen science games.

\chapter{Background}

\section{Citizen Science}

A citizen scientist is ``a volunteer who collects and/or processes data as part of a scientific enquiry." \cite{silvertown2009new} The roots of citizen science date back to the beginnings of modern scientific exploration, where two centuries ago, science was primarily performed as a hobby \cite{silvertown2009new}. With modern communication and the Internet, the number of potential citizens available to do science has grown drastically. Guidelines for scientific research still apply to citizen science projects: the data collected must be validated, the methods of collecting data must be standardized, and volunteers must receive feedback on their contribution \cite{silvertown2009new}.

Citizen science projects have had remarkable success in advancing scientific knowledge \cite{bonney2009citizen}, especially in the bioscience community. These projects often fall into two categories: obtaining data to study large-scale patterns across nature \cite{bonney2009citizen}, or using citizens ability to analyze researcher-collected data that is computationally expensive \cite{cooper2010challenge} \cite{canthepower} or simply too difficult for computers to complete accurately \cite{milkyway2014}.

\subsection{Games With a Purpose}

There are tasks which are trivial for humans but continue to challenge sophisticated computer programs. Traditional approaches to solving this problem focus on improving artificial intelligence systems \cite{gwap}. ``Games with a purpose" \cite{gwap} advocate the constructive channeling of human brainpower through computer games. The Google Image Labeler \cite{googleimagelabeler} is an example of a game with a purpose where users provide meaningful labels to images on the internet, but the game is also fast-paced and competitive. Many games with a purpose avoid using computer vision techniques that do not work well and instead present players with a form of entertainment. ``People play not because they are personally interested in solving an instance of a computational problem, but because they wish to be entertained." \cite{gwap}

The authors presenting ``games with a purpose" propose that the most important aspect of these games is that they are entertaining \cite{gwap}. Even small changes and modifications of the user interface design could influence the enjoyability of these games. The primary objective of games with a purpose is to generate results, whatever they may be. In game with a purpose, throughput could be defined as the number of game objectives completed per human-hour. Games with a higher throughput should be preferred, but it is important that a game is ``fun" as well \cite{gwap}. No matter how efficiently players can solve a game with a purpose, ``fun" is what convinces players to continue playing.

\section{Games}

\begin{quote}
Games are a nascent and complex medium, one which incorporates many previous forms. A single game might include painting, music, cinematography, writing and animation. If that weren’t enough, video games represent an unprecedented collaboration between creator and consumer. We abdicate authorial control to our players and get … something. We’re not quite sure what yet, but we know that it has potential. To many, interactivity seems to be the most important medium of the 21st century. \cite{swink2009game}
\end{quote}

Video games, like the hardware they exist on, have evolved significantly since their birth. More powerful hardware have engaged players with a complex mixture of audio, video, and tactile experiences \cite{atanasov}. Video games are an interesting medium of expression because they encompass so many aspects from a variety of disciplines. Artist disciplines like graphic design, animation, and sound design are expressed with more technical disciplines like computer graphics and computer science concepts. Unlike specifically broadcast mediums like radio, television, newspaper, or books, video games have an added layer of interactivity \cite{schell2008art}.

Designers of novels, television, and other linear entertainment all stress the importance of user experience \cite{schell2008art}. Readers of novels do not influence the experience of the novel, but the novel instead controls the experience. In video games, the distinction betwen the game itself and the experience is much clearer because there is more control by the player \cite{schell2008art}. The player can control which events happen, the pace of the events, and the randomness they may encounter. Feedback is important in games because it reinforces clues about what effects the player's input have.

\begin{figure}
\centering
\includegraphics[width=100mm]{images/relationships.pdf}
\caption{The tight relationship between user input and system output is the foundation of great game experience}
\label{fig:relationships}
\end{figure}

While video games are interactive, it's important to note that they, like television or novels, are a tool. They are a means to an experience, Schell is very clear that the experience is completely separate from the game itself \cite{schell2008art}. It is easy to say that the game is the experience because it is real and it exists. The player and game are real, but the experience is imaginary and all games are judged by the quality of this experience because it is the reason that people play games \cite{schell2008art}.

Unfortunately for designers, there is no silver bullet of design. Psychology itself focuses on the measurable, repeatable, controlled results of experimentation but treats the mind as a black box \cite{schell2008art}. There is no objective way to design a game that exhibits a specific experience. Makers of video games (or any entertainment for that matter) can only focus on what seems to be true as opposed to what is definitely true \cite{schell2008art}.

\section{Aesthetics}

``Visceral design is the difference between a high aesthetic design and one that feels infused with soul." \cite{brown2013how} Game and mobile app designers heavily depend on visceral design to create experiences that resonate with the player \cite{brown2013how}. It's about making connections that just ``feel" right. It isn't about just one design choice or mechanic, but a series of overarching interface decisions that acheives a feeling of contentment \cite{brown2013how}. According to the authors, the key to creating visceral design is to focus on feedback loops and essential user flow mechanisms \cite{brown2013how}.

Donald Norman's seminal book, The Design of Everyday Things, proclaims that the most pleasure is attributed to extreme usability \cite{norman2002design}. Later, in Emotional Design, Norman admits that aesthetics create an emotion as essential to the user experience as extreme usability \cite{norman2007emotional}.

Donald Norman breaks aesthetics into three design paradigms: visceral, behavioral, and reflective design \cite{norman2007emotional}. Visceral emotions are the lowest level of emotion--quick judgements that determine whether an experience is good or bad, safe or dangerous. Next, the behavioral level interprets experiences as they happen, whether they are pleasurable and effective. Finally, Reflective emotion is the feeling of self-image and satisfaction that one perceives when remembering an experience. Visceral design can most easily be mapped to appearance, behavioral design to the pleasure and effectiveness of use, and reflective design to the self-image, personal satisfaction, or memories created \cite{norman2007emotional}.

These three designs directly influence human emotion and cognition, which Norman continues to describe as inseperable \cite{norman2007emotional}. Aesthetically pleasing objects enable one to perform better. Scientific studies have again and again refined logical choices and explanations while very few take emotion into account \cite{norman2007emotional}. Norman argues that cognition, the logical, rational side of the brain has equal importance with emotion, or how you feel, how you behave, and how you think. Norman says, ``Emotion makes you smart. Emotion is always passing judgments, presenting you with immediate information about the world. Here is potential danger, there is potential comfort; this is nice, that is bad." \cite{norman2007emotional} The cognitive and the affective sides of the brain work together to determine one's satisfaction of a situation. ``The cognitive side interprets and makes sense of the world around you while emotions allow you to make quick decisions about it." \cite{norman2007emotional}

\section{Game Feel}

``The aesthetic sensation of control is the starting experience of game feel." \cite{swink2009game} This is the pure feeling of enjoying interacting with an interface and having it respond to input---the visceral design component. Experiencing game feel as skill is the process of leanring. This is why some controls feel intuitive and why some control schemes are easier to learn than others.

Game feel is positive feedback from the experience of video games \cite{swink2009game}. Even as game designers, there is no agreed upon defintion for the language to describe game feel. A ``good-feeling" game is one that let's players do what they want without requiring extensive throught process. ``It is to video games what exists in external activities--the aethetics of driving cars, riding bikes, and so on--but nowhere is it so refined, pure, and malleable." \cite{swink2009game}

Game feel is composed from three parts: real-time control, simulated space, and polish \cite{swink2009game}. These ``building blocks of game feel" translate interactions in to experiences. Figure \ref{fig:blocks} demonstrates the connections between these concepts. 

Real-time control is a specific system of interactive where the player intent is transformed into action which the player interprests from the systems output. The user can then percieve the changes and formulate a new action. As players interact with a game in real-time, the correction cycle compares a users actions with the perceptions of changes in the game world. As a player intends to complete an action, they use the games controls and measure the effects to understand how to reach their goal. The correction cycle in figure \ref{fig:correction} separates the levels of intent in real-time control systems \cite{swink2009game}. Though players have a final goal of finding the princess, the correction cycle operates on the level of the avatar moving through the game world. 

Simulated space refers simulated physical interactions in virtual space, perceived actively by the player \cite{swink2009game}. These interactions give meaning and context to the motion and physicality of the objects in space. Players interacting in a simulated space feel that their actions have consequences. They are a frame of reference and gives us the tactile, physical sense of interacting with virtual environments in the same way we interact with our everyday physical spaces. When a player intends to complete an action in the game, a simulated space returns immediate feedback of their action.

Polish refers to the impacts of animations, sounds, particles, and camera shake. These important effects give clues about what type of interaction we are having with game elements and what physical characteristics they are assuming \cite{swink2009game}. Polish is an effect which emphasize or bring clarity to the underlying simulation. Polish effects are only effects that artificially enchance interactions in the game without modifying the underlying simulation and control. Examples are particle effects, crashing sounds as cars collide, camera shake to emphasize a weighty impact. ``This is separate from interactions such as collisions, which feed back into the underlying simulation." \cite{swink2009game}

Many different polish effects can enhance the perception of the game interaction \cite{swink2009game}. ``Juicy" game design borrows heavily from this definition of polish.

\begin{figure}
\centering
\includegraphics[width=50mm]{images/correction.pdf}
\caption{The correction cycle separates the levels of intent in real-time control systems}
\label{fig:correction}
\end{figure}

\begin{figure}
\centering
\includegraphics[width=150mm]{images/blocks.pdf}
\caption{Translating game feel into experiences}
\label{fig:blocks}
\end{figure}

\section{Juiciness}

From an influential article published in Gamasutra, ``A `juicy' game feels alive and responds to everything you do--tons of cascading action and response for minimal user input. It makes the user feel powerful and in control of the world, and it coaches them through the rules of the game by constantly letting them know on a per-interaction basis how they are doing." \cite{gamasutra}

The goal of ``juicy" effects is to convey some property about an object or game state by offering feedback clues about how that object interacts with other objects, user input, or its environment. ``Juicy" effects create the difference between a scene of a car starting from stratch and gaining speed and a car screeching and kicking up dust as it speeds away. The car may accelerate at the exact same pace in the two scenarios, but one is loaded with ``juicy" effects that enhance the perception of the experience.

``Juicy" design is aesthetics as much as it is about the experience of playing. Games like Peggle and Bejeweled hark back to the audio-visual bleeps, bloops, flashes of original arcade games, and ``it has to be immediate." \cite{popcap2012} When you're doing well in a ``juicy" game, you don't need to keep your eyes on the score--the game is rewarding you directly through the feedback loop. ``It's not about manipulating behavior, it's about rewarding the stuff that's good for game progress." \cite{popcap2012} A ``juicy" game's appeal doesn't end if when the player reaches the end--simply experiencing the game is fun.

\begin{figure}
\begin{center}
\includegraphics[width=80mm]{images/ninja.pdf}
\caption{With one slice of Fruit Ninja, fruit sections, pulp, juice, and particles all exhibit second order motion}
\label{fig:ninja}
\end{center}
\end{figure}

``Juicy systems reward the player many ways at once. When I give the player a reward, how many ways am I simutaneously rewarding them? Can I find more ways?" \cite{schell2008art} The interface is meant to be more than just a means of communication of information, the interface should be alive, engaging, powerful, and interesting. ``Juicy" interfaces often exhibit plenty of second order motion; that is, motion that is derived from the action of the player." \cite{schell2008art} When you move your finger across the touch-based ``Fruit Ninja", your finger turns into the sharpest knife in the world without a visual representation of a knife. In figure \ref{fig:ninja}, sections of fruit fly in opposite direction and fruit pulp splatters on the wall in a ``juicy" display of second order motion. 

The user deserves to play and explore the possibilites in the ``juicy" interface whereas the ``dry" interface quickly becomes a chore. ``Juiciness" is the combination of satisfaction and empowerment contributing to an overall positive experience \cite{atanasov}.


\subsection{Animation}

\begin{figure}
\begin{center}
\includegraphics[width=80mm]{images/bounce.pdf}
\caption{The changing shape of the ball as it bounces creates a realistic perception of a bouncing ball, even though the animation doesn’t directly emulate a bouncing ball}
\label{fig:bounce}
\end{center}
\end{figure}

The basic ``principles of animation" were developed by the original animators of Walt Disney Studio, Frank Thomas and Ollie Johnston, during the 1930s \cite{thomas1981disney}. The ``principles of animation" originated the widespread use of concepts like ``squashing and stretching", ``slow in and slow out", ``exaggeration", and ``appeal". 

``Squashing and stretching" gives the illusion of weight and volume to a particular animated effect. It illustrates something fascinating about animation--it is much more believable to exaggerate animations rather than attempting to perfectly replicate real physical properties \cite{swink2009game}. For example, when the bouncing ball animation reaches the ground, viewers are convinced and interested when the ball squishes to almost nothing at the bottom proceeded by stretching when the ball is mid-air. Martin Jonasson uses squashing and stretching as a subtle effect to give life to collisions and interactions in their breakout game \cite{juiceitorloseit}.

\subsubsection{Tweening}

``Slow in and slow out" refers to a specific type of animation that attempts to model accelerations and decelleration. Short for inbetweening, ``tweening" is the process of generating animation frames between two states, giving the appearence of evolution from one state to the next. The process dates back to traditional animation when the head animator would draw the keyframes and have the inbetween frames completed by their assistant. Computer animators use tweening to complete animation between certain desired key frames in animation, or certain states in game design. Tweening is a function of translation over time of a scalar, but vectors can be broken down into multiple scalar values \cite{penner2002robert}. As Martin Jonasson states, ``you can't always use tweening, but it's dirt easy to implement and it feels luxurious." \cite{juiceitorloseit} Most physical scalar values in ``juicy" video games are ``tweened" in some way.

In the physical world, objects infrequently changes states instantly. Whether the change be translation, rotation, colors, or opacity, tweening gives liveliness to motion and make computer elements interesting to watch. Tweening is perfect for game animation because with the quick change of a tweening function, different elements can exhibit completely transitions invoking a different emotional response to its behavior.

\subsection{Visual Effects}

Where animations dictate how objects move in context of their simulated space, visual effects highlight the interaction between objects \cite{swink2009game}. Usually, visual effects appear only momentarily such as sparks flying off the bottom of a car or a crate shattering into an array of splinters. Visual effects can also be caused by an object, though it is not the animation of the object itself. Many sword fighting games employ this effect. A streak of light will follow a sword to emphasize the speed and strength of the character wielding it. In the ``juicy" breakout clone by Martin Jonasson, screen flashes and shaking are used to emphasize the weight of a collision \cite{juiceitorloseit}.

These effects encompass particle effects too. Particle effects are typically temporary indicators of movement or interaction or a specific quality of an item. Smoke and fireworks are common particle creations, and the motion of the particles is much more important than their color or shape. The motion is how players associate meaning with the particles \cite{swink2009game}.

\subsection{Sound Effects}

Sounds effects are repeatable sounds that players can associate with particular interactions in a game. Often a range of sounds will associate with an interaction to keep the players from hearing the exact same sound over and over \cite{swink2009game}.

\chapter{Related Works}

\section{Citizen science}

The challenge of designing scientific discovery games is that interaction design must be optimized for suitable human interactions in the exploration process while still respecting the scientific requirements \cite{cooper2010challenge}. Fold-it is able to design new drugs by leveraging the creative side of humans brains to organize proteins. The game originally attracted the biochemist community, but the creator, explains during Science Friday on Public Radio International, ``most biochemists quickly left the game because they were pummeled by ordinary people who had incredible spatial recognition skills." \cite{sciencefriday2014}

\subsection{Fold-it}

Fold-it coins the term scientific discovery games \cite{cooper2010challenge} to describe their system. Scientific discovery games are differentiated from general citizen science games because they focus on the problem solving ability of humans to solve computationaly difficult problems \cite{cooper2010challenge}. Fold-it incorporates many traditional aspects of game design; the highlight of Fold-it's design are the use of introductory levels to draw newcomers and explain the mechanics and the requirement that the game still be fun. 

Complex graphical structures are scientifically necessary to illustrate protein structures, but they must also promote human ability to understand those complex structures \cite{cooper2010challenge}. The visualizations in scientific discovery games have several requirements: they must ``reflect and illuminate the natural rules of the system" \cite{cooper2010challenge}, ``manage and hide the complexity of the system" \cite{cooper2010challenge}, and be ``approachable by players" \cite{cooper2010challenge} who have no knowledge of the scientific problem. They should be inviting and fun, not reminiscent of high school science textbooks. In order to make the game approachable, the protein has a bright, cartoonish feel. 

The visualizations of the game are mostly cohesive, but it obvious that the aesthetics were not the priority when optimizing the players first interactions with the game. For instance the login page has an empty box which doesn't seem to contain any information, but still blocks a significant portion of my vision. There are some icons indicating my options, but they aren't necessarily carefuly chosen. The ``Play Offline" icon (two computers with a do-not-enter sybmol) differs significantly from the ``Play Online" button which is a smiley face.

The interactions within the puzzle are intutive and simplified. The creators of Fold-it emphasized interactions that are sufficient to explore, yet intuitive and fun \cite{cooper2010challenge}. Fold-it prototyped games that used sliders to indirectly manipulate the protein, but users found them unintuitive \cite{cooper2010challenge}. Fold-it highlights ``touchability"--the feeling of grabbing and manipulating protein structures with my mouse, as opposed to rearranging sliders for the same effect \cite{cooper2010challenge}. Clicking and dragging with the mouse produces real-time feedback and enables the corrective cycle. While rearrange the shape of the major backbone, there are large pulsing red shapes that indicate when the structure creates physically impossible ``clashes" or ``voids" which warrants correction.

While the sound effects associated with Fold-it aren't particularly pleasing or reminiscent of proteins or this metaphore, they are indicitive of my current state. Sounds accompany different states including dragging and pulling certain elements or using one of the predetermined procedural ``shake" or ``wiggle" functions. After completing a tutorial step, a simple ``Congratulations!" message flashes quickly. The primary motivator to complete certain shapes depends entirely on watching the score increase and decrease the shape of the protein is modified. This is intentional to ``direct players towards the solution" \cite{cooper2010challenge} and is the only significant indicator whether some seemingly-trivial motions of the protein are good or bad for its overall structure.

Fold-it's tutorial are designed to teach non-experts by introducing concepts one level at a time. It is helpful for those who want to just jump right into the puzzles and start folding proteins. The first thing you can do in Fold-it is play with the structure of a protein and begin folding. The higher-level concepts are omitted. There are text bubbles that pop up and guide your mouse to the right tools and options.

Fold it also includes more traditional game elements to keep serious players solving puzzles. The leaderboards are noted by Cooper to be highly compettive \cite{cooper2010challenge}.

\subsection{The Milky Way Project}

The Spitzer Space Telescope gathers infrared light from deep within the Milky Way \cite{milkyway2014}. Scientists have been collecting data for the last 20 years trying to understand how certain diffuse materials collect and create recurring structures and patterns. Users are shown bit-sized images of the Milky Way and provided tools to identify galaxies, star cluters, and egos. Image recognition is a common citizen science task because computationally identifying images is still a field of constant research. Fortunately, humans are surprisingly good with our eyes to make identifying irregular objects easy, but unfortunately not every person can identify each image precisely. The Milky Way project gives the same image to many people and compares the results to settle on a ``correct" identification. Scientists at the University of Oxford leverage the patterns identified by ``untrained" citizens to train their machine learning algorithm, Brut, which is then trained to discover ``bubbles" in the Milky Way \cite{milkyway2014}.

The interface itself is very simple and it is built for a simple task--identifying certain shapes in images. The simplicity of The Milky Way Project should not be overlooked; they have narrowed down the problem scope to an easily digestible size which makes recognition quick. The tutorial system allows users to practice on an image or automatically simulates mouse clicks that complete the tutorial for them. It is simply, but short and effective because it imitates exacty what the player will be doing in the game. There is no description of what a star cluster or a bubble represents without navigating to the ``Science" page. The primary motivation to complete these tasks is simply to partake in science, though images of stars are not displeasing.

\subsection{Cure to Play: Genes in Space}

Genes in Space is interesting because it applies a much more traditional game atmosphere designed around abstrating out the core scientific element. The game is based on the fundamental structure of DNA chromosomes. Cancer cells exhibit certain genetic faults--changes in A, C, G, or T--that result in huge changes in parts of chromosomes known as copy number alterations. (Can the power of the public help personalise cancer treatment?) Copy number alterations can help predict the course of the disease, but it is difficult for computers to identify copy number changes. The human eye is great at detecting these shifts.

Software to identify these differences in DNA micro arrays exists, but it is unable to identify as many patterns as humans can \cite{canthepower}. Software has proven to take significantly identify this problem, but without results published from Genes in Space, it is impossible to compare the two methods \cite{curtis2012genomic}.

Developers from Google, Amazon, and Facebook created the game during a weekend known as GameJam during March 2013 \cite{canthepower}. The DNA microarrays containing defects are translated into ``routes" which your spaceship flies through to collect Element Alpha. On top of mapping your routes (the primary function of identifying copy number alterations), there are asteroids to destroy, ship upgrades to buy, and actual control of the spaceship to keep gamers entertained.

Genes in Space was conveniently developed for mobile, but over only one weekend. The weekend was focused on implementing mechanics and essential game features while the aesthetic qualities lagged. At the time of this report, the game is unplayable on iOS. 

\section{Juiciness in Games}

Simeon Atanasov studied the effects of ``juicy" design in a simple prototype, but not of the citizen science genre \cite{atanasov}. His conclusions site the importance of every aspect of the game; mechanics, simulated-space, and real-time control, as well as ``juiciness". He claims that a player's opinion of ``juiciness" can vary from the next player and that the interpretation of ``juiciness" is based on the interpreted experience. He believes the biggest strenght of ``juicy" design is that it ``becomes a term tightly connected within a particular design, turning from a vague description to a way of keeping a concentrated mind over what we want to achieve as designers." \cite{atanasov} It keeps designers focused on the parts of game design that enable good feedback channels as well as putting a natural language word on this concept.

\section{E. coli Clustering}
\subsection{Pyroprinting}
\label{pyroprinting}

\begin{figure}
\begin{center}
\includegraphics[width=80mm]{images/dna.pdf}
\caption{Two sets of seven DNA sequences from unique copies of the 16S–23S (green) and 23S–5S (blue) ITS regions}
\label{fig:dna}
\end{center}
\end{figure}

Dr. Black and Dr. Kitts of the Cal Poly Biology department have developed a library-dependent technique for comparing DNA fingerprints of bacterial isolates--specifically the identification and classification of E. coli strains. Pyroprinting refers to this library-dependent method of maintaining a database of pyroprints to represent genotypic information of bacterial isolates. Each pyroprint is generated from DNA of specific, highly variable regions in a microbial genome \cite{JanSoliman} call Intergenic Transcribed Spacers (ITS). Pyroprints are generated from two ITS regions around the rRNA genes in E. coli \cite{JanSoliman}. Between the 16S and 23S genes is the ITS region 16S-23S and between the 23S and 5S genes is the ITS region 23S-5S. There are typically seven copies of the rRNA operon in the E. coli genome and to generate a pyroprint all seven ITS loci are amplified and sequenced in a single reaction to maximix potential discrimination between strains \cite{JanSoliman}. There is variation between the 23S-5S region and the 16S-23S region in all seven loci of an E. coli sample. Pyrosequencing a single E. coli sample results in two pyroprints per isolate: one for the 16S-23S region and one for the 23S-5S region. A pyroprint is then represented as a vector of floating point values corresponding to the intensity of the reaction for each nucleotide released during the pyrosequencing reaction. Despite the relatively short sequence length from the pyrosequencers, the potential of discriminating between strains is maximized \cite{JanSoliman}. Pyroprints cannot be used to determine specific DNA sequences they are generated from, but rather they are a unique pattern analogous to a fingerprint of a microbe.

\subsection{CPLOP}
\label{cplop}

Pyroprinting is supported by a web-based database application that stores, retrieves, and analyzes isolates with their associated pyroprints as well as other relavant information, Cal Poly Library of Pyroprints (CPLOP) \cite{JanSoliman}. CPLOP is maintained by collecting samples from the environment, creating the pyroprints with pyrosequencing, managing a pyroprint library, then creating meaningful connections between samples. As the library grows to an appropriate size, unknown samples can then be compared with the database to determine their source. 

MST must include a method of determining similarity between microbes. In the case of CPLOP, there must be a way to compare similarity between pyroprints. CPLOP fulfills this requirement with a Pearson correlation between two pyroprint vectors of the same region \cite{AldrinMontana}.

In order to perform MST with our library of pyroprints, there must be a formally defined similarity between pyroprints. Cal Poly statistics student, Diana Shealy, conducted a study to determine pyroprint comparisons into three categories: definitely similar, definitely dissimilar, and reasonably similar \cite{DianaShealey}. She defined two values $\alpha$ and $\beta$ corresponding to the threshold for similarity and dissimilarity, respectively. A Pearson correlation score between $\alpha$ and $\beta$ might mean that a pair of pyroprints are similar, but it could also mean that a false negative or false positive has occured. Diana was able to determine thresholds where the number of false negatives are 1\%, 5\%, and 10\% at 0.9953, 0.9941, and 0.9915 respectively.

CPLOP \cite{JanSoliman} set the $\alpha$ and $\beta$ thresholds for CPLOP to be 0.995 and 0.99 respectively. These thresholds apply to both the 23S-5S and the 16S-23S regions.

If an isolate has several pyroprints for each ITS region, the similarity is based on a pairwise aggregation of the pyroprints for each region \cite{JanSoliman}.

\subsection{Strain Identification}
\label{strain-identification}

Pyroprints represent the genotype of each ITS region and this makes pyroprints an appropriate vehicle for determining similarity in genotypes between pairs of bacterial isolates. A strain a conglomeration of similar isolates where each isolate in the strain is of some similarity to all other isolates in the strain. The task of creating strains is then determining which isolates are similar which is representative of computationally determining clusters from our dataset. The definition of a strain can fit the definition of a cluster using the Pearson correlation as a similarity metric. The interpretation of each cluster is parallel to the concept of the bacterial strain, though there is no guarantee that each cluster correlates directly to a strain. Each cluster may represent a strain or at least closely resemble one.

\subsection{Agglomerative Heirarchical Clustering}

Partitional clustering algorithms like K-means clustering work by developing centroids and then assigning data points based on the distance from centroids \cite{Liu}. Partitional algorithms work best when there is an estimate number of clusters, but there is no estimate of number of strains when clustering begins. As new data is added to the database it becomes more difficult to estimate the number of clusters and K-means becomes less reliable. A clustering algorithm based on agglomerative hierarchical clustering is utilized by CPLOP due to no apriori knowledge of the number of partitions \cite{JanSoliman}. In hierarchical clustering, each item starts in it’s own cluster. Items are groups by a similarity to other clusters by combining two clusters. The two clusters with the highest similarity are grouped first, then the next more similar clusters, and soon until the items are in a single cluster. The result is a dendrogram. Items clustered earlier are more similar and items clustered later are less similar, and there is still some similarity score that connects each cluster. Hierarchical clustering solves the problem of an unknown number of clusters by allowing any threshold of similarity and the resulting clusters.

\chapter{Implementation}

The goal of POOP-SNOOP is to provide an interface that motivates non-scientist citizens to solve puzzles to further the knowledge base created by the Biology Department at Cal Poly, their work gathering and pyroprinting E. coli, and analyzing that information with various tools including CPLOP. 

From the beginning, the goal was to introduce exaggerated juiciness into a citizen science game. The core mechanic of POOP-SNOOP is simple and there is plenty of room for modification many possible directions. A web browser based game was chosen to maximize possible participation; the prototypes are both created using vectorized graphics with help from a basic tweening library. 

The most popular citizen science projects are hosted on Zooniverse \cite{borne2011zooniverse}, who indicate that over one million participants have helped solve science with their website although Zooniverse is a collection of dozens of games from a variety of research projects and designers \cite{borne2011zooniverse}. On the other hand, Fold-it requires an installation and claims 57,000 have participated \cite{cooper2010challenge}.

\section{Mechanics} 

The puzzle itself is very reminiscent of the spreadsheet biologists originally manipulated to visually identified strains.

\subsection{Modified self-similarity matrix}

The original spreadsheet consisted of E. coli isolates organized into a self-similarity matrix of isolates in each row and column. The intersection between a row and a column compared the two isolates; a self-similarity matrix to compares a dataset to itself. To identify strains within a dataset, both the 16S-23S region and the 23S-5S regions of each pyroprint. In the standard self-similarity matrix, there are two comparisons between each different member of the dataset. Each E. coli isolate must be compared on both the 16S-23S region and the 23S-5S region to be considered in the same strain, therefore by modifying the self-similarity matrix to account for these two different comparisons the modified self-similarity matrix satisfies our requirement.

Every E. coli isolate in our dataset is compared to each other dataset twice--one comparison between the 23S-5S region and one comparison between the 16S-23S region. E. coli isolates are compared to themselves once, but must be in the same strain as themselves by identity. 

The self-similarity matrix must maintain the same order and length in both dimensions. An E. coli isolate is always compared to itself along the diagonal.

The comparison between two regions of DNA in an isolate is represented by a value between 0 and 1. The threshold for strain similarity is a comparison value of 0.95 or greater, and a strain can only consist of E. coli isolates that each compare about the strain similarity threshold on both regions of compared DNA. See section \ref{strain-identification} for more detail.

\subsection{Identifying strains}

Originally, identifying strains was the process of manually swapping two columns then carefully swapping the corresponding two rows to maintain the self-similarity matrix. Strains emerge when perfectly square clusters of DNA comparisons align along the diagonal. Metaphorically this indicates that each isolate in the indicated rows (the columns are the same isolates) could be in a strain together.

The goal of POOP-SNOOP is to encourage players to try different organizations of the puzzle to find the best possible strain. The possible solution set size is n-squared, which could be easily traversed algorithmically. This game is completely unnecessary, but an example of a potential scientific discovery puzzle game.

\section{POOP-SNOOP}

\begin{table}
\begin{center}

\begin{tabular}{|>{\centering}p{3cm}|>{\centering}p{5cm}|>{\centering}p{5cm}|}
\hline 
Category&  Juicy&  Juiceless
\tabularnewline
\hline 

Puzzle initialization&  Puzzle descends into place&  No animation
\tabularnewline
\hline 

Mousedown row/column&  Bouncing animation, simulated depth&  Black outline
\tabularnewline
\hline 

Mouseover row/column&  Indicator animation pointing in direction of travel&  Black indicator
\tabularnewline
\hline 

Dragging row/column&  Other rows slide into new position&  No animation
\tabularnewline
\hline 

Mouseup row/column&  Animation highlighting scored cluster, background effects&  No animation
\tabularnewline
\hline 

Tutorial text&  Slides and fades, fonts, positioning&  No animation
\tabularnewline
\hline

Tutorial slides& Various animations per slide&  No animation
\tabularnewline
\hline

\end{tabular}

\caption[``Juicy" elements of POOP-SNOOP prototypes]{Summary of differences between ``juicy" and ``juiceless" POOP-SNOOP puzzle and tutorial.}
\label{table:juice}
\end{center}
\end{table}

The essential mechanic of POOP-SNOOP is essentially reordering objects in a single dimensional list while observing their relationship with its local cluster. This data is easily represented as a spreadsheet-like grid such that players can associate with the datastructure.

Unlike many games including Fold-it \cite{cooper2010challenge} or Juicy Breakout \cite{juiceitorloseit}, there isn't a specific physical metaphor for a real object that POOP-SNOOP is emulating. A spreadsheet can be thought of as a grid, or a matrix, but other than in computer applications people do not often deal with these structures. Still, in order to make this game enticing, design hints needed to be incorporated.

\subsection{The Grid}

\begin{figure}
\begin{center}
\includegraphics[width=150mm]{images/grid.pdf}
\caption[Differences between game layout in each prototype]{The left image is the ``juiceless" prototype. Circles indicate the current solution, black outline indicate the selected row/column pair. On the right is the ``juicy" prototype with the selected row/column pair.}
\label{fig:grid}
\end{center}
\end{figure}

Spreadsheets remind players of work and many people associate games in a completely different category than work. To incorporate the grid structure of spreadsheets without the work, the design followed a clean grid structure. Elements in the grid are the same size to represent that they are essentially representations of the same objects in our puzzle. The simulated space of POOP-SNOOP should imply a table or flat plain surface which doesn't interfere with the puzzle. In figure \ref{fig:grid}, each comparison between isolates is represented by a block--light gray is matching, dark gray is not matching.

Though it was originally a bit sarcastic, Petri Puhro \cite{juiceitorloseit} suggested personality by adding faces and adjusting their eyes to elicit emotional response. In figure \ref{fig:grid}, comparisons in the ``juicy" prototype contributing to the solution are given a smiling face. The faces of the boxes are visible when they are part of a larger cluster. The intent is that the player would be encouraged to put as many ``happy" blocks together to find the largest solution. In the non-juicy version, dark circles indicated cells that were part of a solution.

\subsection{Interaction}

Players need to reorganize the grid easily and play with its orientation. The most fundamental metaphor for this in computer applications is clicking and dragging. If a computer user were to click an icon from the desktop and drag it, they could change its position. Unfortunately, the relationships in the grid require that the entire row and column move with it. Without explicitly connecting them in some way, the intent is that players would be able to observe the results of their action as part of the real-time feedback loop and learn the mechanic this way. As players click and drag, the grid reorganizes around them to immediately reflect the implications of their decision.

\begin{figure}
\begin{center}
\includegraphics[width=100mm]{images/chevron.pdf}
\caption[Differences between chevrons in each prototype]{On the left, ``juicy" chevrons when unselected and selected. On the right, ``juiceless" chevrons when unselected and selected.}
\label{fig:chevrons}
\end{center}
\end{figure}

The ability to drag squares should be made obvious to players. The pieces of the grid are all important to the current and possible solutions, but only the pieces along the diagonal are interactive. To exhibit this characateristic, chevrons pointing in their potential range of travel were added. The chevrons intended to manipulate a physically rough surface or small bumps and the arrows indicated movement. In figure \ref{fig:chevrons}, the differences between the ``juicy" chevron and ``juiceless" chevron are demonstrated; the juicy chevron is animated.

In ``juicy" POOP-SNOOP, each box has a shadow and pops up off the plane of the rest of the grid when moused over. A simple opacity and shadow gives the pieces of the puzzle depth when rows intersect slightly as players rearrange the puzzle. While moving the puzzle, players can see other pieces through theirs, adding to the illusion of depth. Overall, it resemebles the metaphore for small, translucent pieces of paper. The rows and cells of the grid bounce up into a hover state as they become activated which is playful, but subtle enough to not be overwhelming. As the player actively rearranges the puzzle, they are given visual hints about their actions because the rows slide gently into their new place, rather than simply appearing in their new position. Because the movement of the rows and columns is unnatural, this behavior intends to reinforce the mechanic for new players. The objects in POOP-SNOOP do not interact with eachother, but slide past one another.

As the boxes of the grid move, their motion is subtley enhanced by stretching and squeezing the direction of their movement. It is especially difficult to see how the puzzle changes when a column and row change orientation.

\subsection{Scoring}

Scientific discovery games exist because researchers do not know the solution to the given puzzle. Players would not normally know the solution of a POOP-SNOOP puzzle, but a message that alerted users when they solved the introductory level. In the nature of ``juicy" design, the intent was that players would be given clues other than a number to indicate their score. The size of the score itself is a very visual representation of score.

Rearranging the puzzle consisted of a mousedown action to enable dragging, dragging the mouse to move the row-column pair, then releasing the mouse to return the grid to a resting state. As the player released their currently selected row, the intention of the game was to respond to that change and convey the current score to the player. Cluster highlighting happened simultaneously with background effects.

Background effects are designed reward positive player behavior with exciting explosions, colors, and motion. The goal of the effects is to encourage players to keep trying new combinations and working towards the best possible solution. The effects were designed to give more feedback for larger, better solutions and less feedback for smaller solutions. Five separated effects consisting of geometric shapes and patterns animated and exploded when dragging was complete and the score was totalled; an example of these effects can be seen in figure \ref{fig:background}. The background effects lasted about two seconds and interrupted gameplay because the puzzle was hdiden. Players were able to click and end the animation early, but it was not made obvious.

\begin{figure}
\begin{center}
\includegraphics[width=150mm]{images/background.pdf}
\caption[Differences between animations in each prototype]{On the left, the animation when completing a drag in the ``juicy" prototype. On the right, the same event in the ``juiceless" prototype.}
\label{fig:background}
\end{center}
\end{figure}

\section{Tutorial}

POOP-SNOOP requires a tutorial. Feedback during the design phase indicated that the scientific knowledge helped players understand why they were essentially reorganizing a grid. The mechanics themselves were confusing and unintuitive to new players. Like other scientific discovery games, the tutorial is the first section of the game that people interact with, but does ``juicy" design have implications here? Each version of POOP-SNOOP was tested with a prototypical tutorial.

The content of the tutorials was designed to be identical; each tutorial had the same illustrations, same text, and the same order. Within the tutorial, players interacted with a small version of POOP-SNOOP. The ``juicy" version of the tutorial also contained the ``juicy" version of POOP-SNOOP and vice versa. The ``juicy" version was given some design thought. The placement of text, color, and overall feel was augmented to measure its effect on learning. One slide from both versions of the tutorial are shown in figure \ref{fig:tutorial}.

\begin{figure}
\begin{center}
\includegraphics[width=150mm]{images/tutorial.pdf}
\caption[Differences between tutorials in each prototype]{On the left and right, an example of the ``juicy" tutorial and ``juiceless" tutorial, respectively.}
\label{fig:tutorial}
\end{center}
\end{figure}

\subsection{Text}

In order to entice players to read the text and digest the scientific concepts, text in the ``juicy" tutorial assumed the metaphor of speech bubbles being recited by an E. coli character. The intent of this design was that players would associate the character as guiding them in the tutorial. Special text effects should encourage player engagement. 

\subsection{Animations}

Both versions of the tutorial had the same content, but the ``juicy" version of the puzzle would often introduce content in a more appealing way. The DNA slide ``spun" into view and the lines indicating a ``match" were animated and bouncing. As POOP-SNOOP was introduced it gently slid down into the player's space, whereas the puzzle simply appeared in the ``juiceless" version.

\chapter{Results}

\section{Experimental Setup}

To understand the differences between the juicy and juice-less versions of POOP-SNOOP and evaluate my prototype, two experiments were executed. 88 Cal Poly undergraduate students and professors anonymously reviewed only the ``juicy" version of the game while 45 students and professors anonymously reviewed only the ``juice-less" version. In a separate experiment, a separate group of twelve Cal Poly students reviewed both ``juicy" and ``juice-less" versions of the game side-by-side while accompanied by a proctor.

The ``juicy" and ``juice-less" prototypes contained exactly the same text, shapes, slides, and order. The intension was to isolate juiciness, and the specific differences were identified in the Implementation Section. In the prototype, the users first completed the tutorial section which led directly into a introductory level for participants to utilize their new skills. When the puzzle had been arranged in the optimal solution, they were told that the best solution had been solved, though they were still allowed to continue playing that puzzle.

\subsection{A/B Testing}

To gather subjects, an opportunity to participants was presented to four classrooms of 20-30 students. More participants were only given the option to participate via email. All participants were given an introduction verbally or via email. The introduction briefly outlined the purpose of citizen science games, and basic instructions to complete. The essential instrutions were embedded in the game itself. 

Within the game, subjects were asked to fill out a anonymous feedback form after the tutorial section and then another anonymous feedback form after playing an introductory level. The anonymous feedback forms used can be found in the appendix.

\subsection{Focus Group Testing}

Ten of the focus group members study a design-related subject, but none of them were involved in a computer science related field. Zero of these participants had played a citizen science game before.

Each focus group member played both versions of the prototype while being observed. They were allowed to ask questions about clarifications and encouraged to speak about their emotional response to the prototype while they played. Half of the subjects played the juiceless version first, and the second half played the juicy version first. Before the subjects were given a prototype, they were told about the nature of citizen science games some details about the origins of the data used in POOP-SNOOP. Zero participants filled out the surveys, each only provided verbal feedback.

\section{Evaluation}

The most significant observations from the A/B testing are:

\begin{enumerate}

\item Users completing the ``juicy" tutorial tended to have higher confidence about solving POOP-SNOOP puzzles.

\item Users completing the ``juicy" tutorial indicated more interest continuing after the tutorial to solve POOP-SNOOP puzzles.

\item Users playing the ``juicy" introductory level found their puzzle less difficult.

\item Users playing the ``juicy" introductory level indicated that the tutorial was more helpful than ``juiceless" players.

\item Users playing the ``juicy" level indicated more enjoyment of POOP-SNOOP than ``juiceless" players. 

\item Frustration with fundamental game mechanics and tutorial unclarity was voiced by more ``juiceless" players than ``juicy" players.

\item More ``juicy" prototype players specifically mentioned encouraging graphics.

\end{enumerate}

From the focus group study, observations concluded: 

\begin{enumerate}

\item The chevrons were helpful in indicating the primary mechanic in both ``juicy" and ``juiceless" prototypes.

\item The ideal POOP-SNOOP would be comprised of features from both the ``juicy" and ``juiceless" prototypes.

\item There was excessive feedback when making changes to the solution to the point that it interfered with solving the puzzle.

\item The incremental ``juicy" feedback regarding different sized solutions was not effective.

\end{enumerate}

\subsection{A/B Testing}

The two surveys given during each A/B testing session, the first after the tutorial and the second after the introductory level. The appendix lists the two surveys distributed. Table \ref{table:tutorialsurvey} is the summary of these qualitative results from the post-tutorial survey. Table \ref{table:gamesurvey} is the summary of quantitative results from the introductory level of POOP-SNOOP. Participants of the ``juiceless" experiment completed 58 post-tutorial surveys and 43 final surveys. Participants of the ``juicy" experiment completed 54 post-tutorial surveys and 66 final surveys. It is not clear why there were not an equal number of surveys completed for each version of POOP-SNOOP, but without any insight into the reasons for the discrepancy none of the results were discarded.


\begin{table}
\begin{center}

\begin{tabular}{|>{\centering}p{3cm}|c|c|c||c|}
\hline 
Question&  Mean Juicy&  Mean Juiceless&  Difference&  p-value
\tabularnewline
\hline 
How confident do you feel about solving a POOP-SNOOP puzzle?&  3.0&  2.3&  1.3&  0.003
\tabularnewline
\hline 
How interested are you in solving a POOP-SNOOP puzzle?&  3.3&  2.5&  0.8&  3.1E-5 
\tabularnewline
\hline 
\end{tabular}

\caption[Post-tutorial survey results]{Comparison of the difference in confidence and interest between players of the ``juicy" and ``juiceless" prototypes.}
\label{table:tutorialsurvey}
\end{center}
\end{table}

All quantitatively measured results from the surveys were statistically significant (p-value $<$ 0.05), the results are listed in Table \ref{table:tutorialsurvey} and Table \ref{table:gamesurvey} and summarized in the beginning of this chapter.

In the ``juiceless" version, 45 of 58 responses cited confusion about game mechanics, purpose, or goal, where only 17 of 54 ``juicy" participants agreed. There may be a connection between making interesting tutorial levels that increases players attention to learning the game or reading text. Players found the ``juicy" version more enjoyable, and 5 of 66 responses specifically mentioned that the ``juicier" graphics were encouraging and helpful, though they had not seen the ``juiceless" version.

Surprisingly, 6 of 43 ``juiceless" players specifically mentioned ``visually clean", ``great interface", ``clean presentation", or praised the visual design. Another astonishing comment from the ``juiceless" survey suggested, ``You need to make the game juicy" and included a link to the talk by Martin Jonasson \cite{juiceitorloseit} that inspired this project. Two other comments in the non-juicy version also mentioned adding color and doing more user testing. These comments were not ordinary, most comments did not mention anything about the design or the ``juicy" design of POOP-SNOOP.

\begin{table}
\begin{center}

\begin{tabular}{|>{\centering}p{3cm}|c|c|c||c|}
\hline 
Question&  Mean Juicy&  Mean Juiceless&  Difference&  p-value
\tabularnewline
\hline 
Did you enjoy POOP-SNOOP?&  3.0&  2.4&  0.6&  0.00147
\tabularnewline
\hline 
Was the puzzle difficult?&  2.4&  3.5&  1.1&  2.8E-7
\tabularnewline
\hline 
Was the tutorial clear and helpful?&  2.4&  1.7&  0.7& 0.00045
\tabularnewline
\hline 
\end{tabular}

\caption[Post-game tutorial results]{Comparison of the difference in enjoyment, perceived difficulty, and understanding between players of the ``juicy" and ``juiceless" prototypes.}
\label{table:gamesurvey}
\end{center}
\end{table}

\subsection{Focus Group Testing}

In every focus group interview, each subject came to the conclusion that the ideal puzzle would contain features of both the juicy and non-juicy versions (overview result 2). When asked to make a choice about which version they would prefer to play another puzzle with, four subjects responded ``juicy" and eight responded ``juiceless".

Subjects tended to prefer the non-juicy version, even if they had been given the juicy version first. Users cited the distracting ``juicy" feedback interfered with their ability to solve puzzles. Users preferred the quicker response of the ``juiceless" version.

Players quickly became annoyed by the overwhelming response of the system for every time they moved the puzzle orientation (overview point 3). The first time the rewarding animations happened, subjects were excited, but it became repetitive quickly. 

The scoring animations were slightly more intense if the score was larger, but no subjects noticed the different between any of the animations (overview point 4). The range of possible solutions was between 2 and 4, restricting the number of animations. There was not enough variation between the response to solutions and the response was too intense for every solution. Participants suggested that the background animation should only reward them the first time they create large solution.

The scoring animations caused player confusion when removing grid blocks that were not part of the solution. Four participants explicitly stated that they thought the puzzle was restarting or changing when the blocks dissapeared and reappared. The emphasis of the solution was inadequate; no players mentioned specifically preferring this behavior.

The chevrons for the diagonal squares were effective in both the juicy and juiceless versions (overview point 1). Early in our prototype process, the grid was simply an array of plain squares with no special markings. There was consistantly a problem with people understanding the core mechanic of POOP-SNOOP: the diagonal drag. Because the grid looked square, subjects always attempted to drag row-column pairs in any direction and simply allowing motion only in the diagonal direction never caught on. Eventually, the idea for chevrons on the diagonal was implemented and to my surprise ten out of twelve focus group participants naturally grabbed the squares along the diagonal and dragged them in the correct direction. The two that didn't understand the dragging motion immediately did not interact with the tutorial phases until they were forced too and seemed to skip the instruction about movement.

The lack of colors in the ``juiceless" often caused confusion in the tutorial because the grays were indiscernible. More subjects interacted with the ``juicy" tutorial when the puzzle first appeared on screen. The three observed participants who played with POOP-SNOOP before the introductory level started had the most success with understanding.

\chapter{Discussion}

\section{The Language of Juiciness}

One of the most important asepcts of designing an experiment around POOP-SNOOP is measuring ``juiciness". The language of ``juicy" design doesn't coincide with players vocabulary. The question, ``is this game juicy?" might invoke some good guesses, but it is not perfect. Without knowledge of their difference, players probably would not discern between the experience of the game and the game itself. But again, is the game what should be measured? If games specifically create experiences, should be both sepearating them? 

The concept of juiciness is defined by the experience created while playing the game. Atanasov \cite{atanasov} states that a juicy experience could be defined as plentiful ``positive emotional feedback", but those words are charged with implications and may skew the results if players were specifically asked about them.

While interviewing players, they understandably found it difficult to describe their emotional state. Intstead, they answered questions about physical conditions of the game. Interviewees pointed out specific aspects they liked whether the smoothness of the interface, the way the chevrons highlighted and bounced, or the flourishes of the background animations. While these emotions can be aligned positive or negative per effect, the overall emotional response is more difficult to analyze.

Putting our game in the hemisphere of citizen science can give a context for our definition of ``positive emotional feedback". In citizen science games, players should feel motivated and encouraged to continue solving puzzles. Asking a player if they would continue playing POOP-SNOOP is not the same measurement as real citizens playing POOP-SNOOP. Because of the nature of my prototype, users were not given the ability to pursue more puzzles. We have sufficiently answered the question that with regards to citizen science games, juiciness can have a substantial impact on player attentiveness.

\section{Juiciness and Game Feel}

Collecting data about playing POOP-SNOOP has demonstrated that ``juicy" is only part of a larger feeling of gamefulness. Like game feel suggests, mechanics play an important role in the quality of a game. Players are susceptible to juiciness' influence while learning about the mechanics of a game, as those who played the ``juicy" tutorial have demonstrated, but there were still plenty of players who mentioned confusion after completing the game. Did their confusion inhibit their ability to answer the survey questions correctly? Would they have been able to answer the survey if they were confused? In this situtation, those results were utilized because there was a tendency for players of the ``juicy" version to be confused less. The experiment shows those those who played the ``juicy" version potentially learned the game better. The goal of ``juiciness" is only to reinforce the mechanics and physicicalilty of the game, but if players cannot understand those concepts, measuring their sense of experiencing ``juiciness" could potentially be misleading.

\subsection{Juiciness and POOP-SNOOP}

Several aspects of playability in POOP-SNOOP potentially interfered with the user's willingness to reflect on subtleties that were not playable aspects. For instance, when the introductory puzzle interrupted a player's freedom, did they take that into account when the measured ``enjoyment"? Enjoyment, interest, and difficulty are not a strictly ``juicy" measurment nor is POOP-SNOOP a perfect example of ``juicy" and ``juiceless" design.

Because some users of the ``juiceless" version praised the clean layout, it's worth mentioning that there may have been extra ``juicy" effects in the ``juiceless" version. By definition, the ``juiceless" version should be a version of the game that has very little design incorporated. While the earlier definition only applies to effects, particles, sounds which are all easily grouped, where does overall design fit in this model?

For instance, the design of the grid was such that it did not resemeble the original spreadsheet that POOP-SNOOP once resided in. Should the ``juiceless" version of POOP-SNOOP reflected that nature? Did incorporating a simple grid structure break my definition of ``juiciness" or is that design part of the fundamental design of the game that cannot be manipulated. By measuring different version of POOP-SNOOP, the consequences of extraneous effects were observed to be as extraneous as possible. Some design challenges that could have been categorized as ``juicy" were included in both versions to isolate the measurement of ``juiciness."

Too much polish is distracting because it makes it difficult to wrap your brain around the physical sensation being conveyed \cite{swink2009game}. Contrast those with little puffs of smoke that come up as Mario slides his feet around. Those are great because they are easy to make sense of. But when every single object sprays stuff everywhere, how does that reconcile with the experience of physical reality?

In contrast, the harmonization of polish and mechanics utilizes effects to support the single impression of physicality \cite{swink2009game}. In a puzzle game, the physicality isn't as obvious as game with the intension of immersion. The phsyical metaphor for this game is weak which limited possible ``juicy" design as well as augmenting some basic physical metaphor with ``juicy" design. As more polish is added, game designers must be careful that their methods do not override the basic game design techniques.

\chapter{Future work}

POOP-SNOOP had many flaws of design. To continue to study the importance of ``juicy" design in citizen science applications it would be pertinent to fix problems of understanding for the users. Instead of introducing so much science originally, the game should be modified such that players can immediately begin playing and interacting with the puzzle. When players begin POOP-SNOOP, they should learn how to solve the puzzle first, then begin to learn the science behind it, if at all. There was very little positive feedback about the scientific portion of the tutorial. Because of the difficulty people had with the tutorial, it should be modified.

Immediate feedback in POOP-SNOOP did not engage users as much as intended. Iterative design would have yieled changes in my design with respect to how the user responds to input. Though this work makes the point of important with every user input and a corresponding output. When players begin moving row/columns and adjusting the solution, there should be feedback; not only when the row/columns are released.

The implementation of ``juicy" positive emotional feedback should be reassessed for future work. Users were too confused be encouraged from bombastic background animations. Work should be done iterating the scope and amount feedback users prefer.

Sound is such an important part of ``juicy" design that it could be just as important as visual feedback of any kind. Sound is especially important because of users senstivity in that aspect as well. Well-crafted tones, effects, and background music could be unparalleled in terms of engagement, unfortunately POOP-SNOOP did not explore sound.

Even the ``juiceless" version of POOP-SNOOP had positive responses to the user interface. Participants praised the ``clean interface" which brings up the question: does a clean interface give the feeling of professionality and seriousness? Is that influencial in a citizen science application? POOP-SNOOP does not attempt to answer these questions.

\section{Long-term Engagement}

POOP-SNOOP only took the tutorial and introductory level into consideration. There is serious value in understanding ``juicy" design's role in long-term engagement, especially in the field of citizen science.



% ------------- End main chapters ----------------------

\clearpage
\nocite{*}
%\bibliographystyle{plain}
\bibliography{Bibliography}
%\addcontentsline{toc}{chapter}{Bibliography}

\begin{appendices}
%\addcontentsline{toc}{chapter}{Appendix}

\chapter*{APPENDIX}


{
   \captionsetup[figure]{labelformat=empty}

   \begin{figure}[H]
   \centering
   \includegraphics[width=115mm]{images/Post-tutorial.pdf}
   \caption{Survey given after the tutorial}
   \end{figure}

   \begin{figure}[H]
   \centering
   \includegraphics[width=115mm]{images/Post-game.pdf}
   \caption{Survey given after the introductory level}
   \end{figure}
}

\end{appendices}

\end{document}
