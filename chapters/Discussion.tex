\chapter{Discussion}

\section{The Language of Juiciness}

One of the most important asepcts of designing an experiment around POOP-SNOOP is measuring ``juiciness". The language of ``juicy" design doesn't coincide with players vocabulary. The question, ``is this game juicy?" might invoke some good guesses, but it is not perfect. Without knowledge of the difference, players would likely not discern between the experience of the game and the game itself. But again, is the game what should be measured? If games specifically create experiences, is measuring the experience the same as measuring the game? 

The concept of ``juiciness" is defined by the experience created while playing the game. Atanasov \cite{atanasov} states that a juicy experience could be defined as plentiful ``positive emotional feedback", but those words are charged with implications and may skew the results if players were asked to describe any ``positive emotional feedback" they experienced during play.

While players were interviewed, they found it difficult to describe their emotional state. Instead, they verbalized their feelings about of the game. Interviewees pointed out specific aspects they liked whether the smoothness of the interface, the way the chevrons highlighted and bounced, or the flourishes of the background animations. While these emotions are aligned to a positive or negative emotion, the overall emotional response is more difficult to analyze. At which point do certain negatives outweight positives?

Putting POOP-SNOOP in the hemisphere of citizen science reveals a convenient definition of ``positive emotional feedback". In citizen science games, players should feel motivated and encouraged to continue solving puzzles. Asking a player if they would continue playing POOP-SNOOP is not the same measurement as real citizens playing POOP-SNOOP. Because of the nature of my prototype, users were not given the ability to pursue more puzzles. We have sufficiently answered the question that with regards to citizen science games, juiciness can have a substantial impact on player attentiveness.

\section{Juiciness and Game Feel}

Collecting data about playing POOP-SNOOP has demonstrated that ``juicy" is only part of a larger feeling of gamefulness. Like game feel suggests, mechanics play an important role in the quality of a game. Players are susceptible to juiciness' influence while learning about the mechanics of a game, as those who played the ``juicy" tutorial have demonstrated, but there were still plenty of players who mentioned confusion after completing the game. Did their confusion inhibit their ability to answer the survey questions correctly? Would they have been able to answer the survey if they were confused? In this situtation, those results were utilized because there was a tendency for players of the ``juicy" version to be confused less. The experiment shows those those who played the ``juicy" version potentially learned the game better. The goal of ``juiciness" is only to reinforce the mechanics and physicicalilty of the game, but if players cannot understand those concepts, measuring their sense of experiencing ``juiciness" could potentially be misleading.

\subsection{Juiciness and POOP-SNOOP}

Several aspects of playability in POOP-SNOOP potentially interfered with players' willingness to reflect on subtleties that were not playable aspects. For instance, when the introductory puzzle interrupted players with exaggerated congratulations, did they take that into account when the measured ``enjoyment"? Enjoyment, interest, and difficulty are not a strictly ``juicy" measurment nor is POOP-SNOOP a perfect example of ``juicy" and ``juiceless" design. Playability is an important fundamental in enjoyment.

Because some users of the ``juiceless" version praised the clean layout, it's worth mentioning that there may have been extra ``juicy" effects in the ``juiceless" version. By definition, the ``juiceless" version should be a version of the game that has very little visual design additions. While the earlier definition only applies to effects, particles, sounds which are all easily grouped, where does overall design fit in this model?

For instance, the design of the grid was as much part of the ``juicy" version as it was in the ``juicless" version. Should the ``juiceless" version have incorporated less visual design, instead of just omitting the particle effects? Did incorporating a simple grid structure break the definition of ``juiciness" provided or is that design fundamental to the mechanics. By measuring different versions of POOP-SNOOP, the intention was to measure the effect of design that was as extraneous as possible, but there are aspects present in both versions that could arguably be ``extraneous". 

Too much polish is distracting because it makes it difficult to wrap your brain around the physical sensation being conveyed \cite{swink2009game}. The little puffs of smoke that come up as Mario slides his feet around are great because they make sense. When every single object has puffs of smoke, how does that reconcile with the experience of physical reality? Though games do not intend to perfectly replicate reality, certain physical sensations make more sense than others. Should ``juicy" affects replicate nature? Or, is the interesting part of ``juicy" effects that they are unexpected? ``Juicy" design fits on this spectrum, but importantly, game designers must take care to associate effects with specific game feedback. Too many extraneous effects are just as confusing as too few.

The harmonization of polish and mechanics support the single impression of physicality \cite{swink2009game}. In a puzzle game, physicality isn't as obvious as an immersive role-playing game. The phsyical metaphor for puzzle games are weak which potentially limits convincing ``juicy" effects. 
