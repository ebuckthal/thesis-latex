\chapter{Introduction}

Though the term ``gamer" still has connotations of an teenager blasting zombies in the basement, gaming has been adopted by mainstream culture \cite{sciencefriday2014}. Bejeweled, Candy Crush, and Farmville are only a few examples of games that have gained widespread acceptance outside of the traditional gamer persona. Computer and video games are a ``form of entertainment enjoyed by a diverse, worldwide consumer base that demonstrates immense energy and enthusiasm for games." \cite{softass} 59\% of American citizens play games, the average player is 31 years old, and 48\% of players are female. Puzzle, board game, game show, trivia, and card games make up 28\% of online games played \cite{softass}.

Leveraging the motivation to play games and humans' ability to recognize patterns, researchers have empowered users to perform citizen science. Examples of this include The Milky Way Project where users identify celestial bodies \cite{milkyway2014} and interactive biology applications such as Fold-it \cite{cooper2010challenge}. This partnership has introduced a more general genre of scientific discovery games which take advantage of human problem solving abilities to solve computationally difficult research problems. 

Because scientific discovery games translate these research problems into games, they rely on many fundamentals of game design including the explanation of game mechanics, the design of introductory levels, and potentially the scientific concepts. More importantly, scientific discovery games' goal is to provide an interface which non-expert players can apply knowledge in a specific scientific domain. \cite{cooper2010challenge} While fun is not the primary objective, citizen science games and other genres can enhance motivation in their game by applying traditional game design strategies. Specifically, this thesis focuses on ``juicy" game design techniques.

In this paper, we define ``juicy" design and it's importance in the genre of citizen science games and scientific discovery games. We also explore several examples of existing scientific discovery games and their ``juiciness." By prototyping and experimenting with two versions of a scientific discovery game, we conclude that ``juiciness" in citizen science games decreases perceived difficulty, increases understanding of fundamental game concepts, and improves enjoyability of citizen science games. We discuss the potential for ``juiciness" to improve player motivation and efficency at citizen science games.
