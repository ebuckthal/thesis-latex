\chapter{Future work}

POOP-SNOOP had many flaws of design. To continue to study the importance of ``juicy" design in citizen science applications it would be pertinent to fix problems of understanding for the users. Instead of introducing so much science originally, the game should be modified such that players can immediately begin playing and dragging the puzzle. When players begin POOP-SNOOP, they should learn how to solve the puzzle first, then begin to learn the science behind it. Because of the difficulty people had with the tutorial, it should be modified.

Immediate feedback in POOP-SNOOP did not engage users as much as intended. Iterative design would have yieled changes in my design with respect to how the user responds to input. Though this work makes the point of important with every user input and a corresponding output, POOP-SNOOP does not adhere to this mantra. When players begin moving row/columns and adjusting the solution, there should be feedback here. Not just when the row/columns are released.

The implementation of ``juicy" positive emotional feedback should be reassessed for further work. Users were too confused to find encouragement from bombastic background animations. Work should be done narrowing down the knowledge and scope of amounts of feedback users prefer.

Sound is such an important part of ``juicy" design that it could be just as important as visual feedback of any kind. Sound is especially important because of users senstivity in that aspect as well. Well-crafted tones, effects, and background music could be unparalleled in terms of engagement. 

Even the ``juiceless" version of POOP-SNOOP had positive UI response. Participants praised the ``clean interface" which brings up the question: does a clean interface give the feeling of professionality and seriousness? Is that influencial in a citizen science application? POOP-SNOOP does not attempt to answer these questions.

\section{Long-term Engagement}

POOP-SNOOP only took the tutorial and introductory levels into account. There is serious value in understanding ``juicy" design's role in long-term engagement, especially in the field of citizen science. 
